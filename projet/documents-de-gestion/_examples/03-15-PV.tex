%!TEX encoding = IsoLatin

%
% Exemple de procès-verbal
% par Pierre Tremblay, Universite Laval
% modifie par Christian Gagne, Universite Laval
% % 2011/01/14 - version 1.4
% modifié par Robert Bergevin, Université Laval
% 24/11/2011
% modifié par Jean-Yves Chouinard, Université Laval
% 2016/01/11
% modifié par Jean-Yves Chouinard, Université Laval
% 2017/01/04
% refonte par Dominique Beaulieu, Université Laval
% 2018-04-04
%

%--------------------------------------------------------------------------------------
%------------------------------------- preambule --------------------------------------
%--------------------------------------------------------------------------------------
\documentclass[12pt]{ULojpv}

% Chargement des packages supplementaires
\usepackage[ansinew]{inputenc}
\usepackage{pifont}


% Definitions des parametres de l'en-tete
\Cours{GEL--1001 Design I (méthodologie)}             % Nom du cours
\NumeroEquipe{99}                                     % Numero de l'equipe
\NomEquipe{Les Bâtisseurs}                               % Nom de l'equipe
\Objet{Procès-verbal}                                 % Nom du document
\SujetRencontre{Revêtement extérieur}             % Sujet de la rencontre
\DateRencontre{2018/03/15}                            % Date de la rencontre
\LocalRencontre{PLT--2708}                            % Local de la rencontre
\HeureRencontre{10h30-12h20}                          % Heure de la rencontre


%--------------------------------------------------------------------------------------
%--------------------------------- corps du document ----------------------------------
%--------------------------------------------------------------------------------------
\begin{document}
\entete
\begin{enumerate}

% nouveau point
\item \textbf{Ouverture de la réunion}
Heure: 10h32

% nouveau point
\item \textbf{Nomination ou confirmation du président et du secrétaire}

\begin{tabular}{@{}ll}
   Président: Jean
   & Secrétaire: Marie
\end{tabular}

% nouveau point
\item \textbf{Adoption de l'ordre du jour}

L'ordre du jour proposé est adopté à l'unanimité.

% nouveau point
\item \textbf{Lecture et adoption du procès-verbal de la réunion du 8 mars 2018}

Le procès-verbal proposé est adopté à l'unanimité.

% nouveau point
\item \textbf{Affaires découlant du procès-verbal}

\begin{enumerate}

\item Suivi de l'étude des options de revêtement extérieur

\begin{enumerate}

\item Types de revêtements possibles

Marie a fait des recherches sur Internet et a consulté son cousin qui travaille dans la construction. Pour recouvrir la niche de Fido, voici les trois options possibles : la peinture à l'huile, le revêtement en polymères et revêtement en acier. La peinture à l'eau n'est clairement pas une option car cela protège mal le bois des intempéries. Marie nous a apporté des dépliants et a produit un tableau synthèse pour faciliter notre compréhension. Marie va élaborer davantage dans les points à traiter.

\item Liste des fournisseurs potentiels

Marie avait aussi pour mandat de trouver des fournisseurs potentiels pour le revêtement extérieur de la niche. Voici les fournisseurs qu'elle a trouvés :

Option 1 : peinture à l'huile

	Fournisseur 1 : BMR Lavoie, 967 vanue Taniata, Lévis. 418-839-0017
	
	Fournisseur 2 : Groupe BMR, 125 rue d'Anvers, Saint-Augustin-de-Desmaures. 418-878-2683
	
	Fournisseur 3 : RONA Quincaillerie Corriveau, 1744 ch. Saint-Louis. 418-683-1901
	

Option 2 : revêtement en polymères

	Fournisseur 1 : Novik, 160 rue des Grands Lacs, Saint-Augustin-de-Desmaures. 418-878-6161
	
	
Option 3 : revêtement en acier

	Fournisseur 1 : Matériaux Ouellet \& Leduc, 15720 boul. Valcartier, Québec. 418-842-2394


Même s'il n'y a qu'un seul fournisseur pour les options 2 et 3, Marie propose d'en rester là pour le moment et selon la décision sur le revêtement choisi, elle pourra poursuivre ses recherches pour trouver au moins un autre fournisseur afin de pouvoir comparer les prix.

\item Plages des prix

Obtenir des prix pour les différentes options était aussi le mandat de Marie. L'équipe convient qu'il serait plus pertinent de traiter ce point pendant la discussion sur le choix du type de revêtement extérieur.

\end{enumerate}

\item Suivi de l'étude des options pour la pose du revêtement extérieur
	\begin{enumerate}
		\item Entrepreneurs possibles
		
		Luc avait pour mandat d'identifier des entrepreneurs qui pourraient procéder à la pose du recouvrement choisi. Voici ce qu'il a trouvé en date d'aujourd'hui :
		
		
		Option 1 : Peinture André Routhier, 815, rue Pierre Maufay, Québec. 418-684-0555
		
		Options 2 et 3 : Rénovation Économique, 1447 rue Frank-Carrel, Québec. 418-681-7272
		

Luc a manqué de temps pour trouver plus d'entrepreneurs et s'en excuse.

		\item L'oncle de Paul
		
		Paul a demandé à son oncle s'il pouvait faire le travail de recouvrement de la niche. Comme il a travaillé plusieurs années comme "jobbeux" en faisant des petits contrats, la réponse est oui.
		
		
		Jean demande à Paul si son oncle a ses cartes de la CCQ et Paul avoue de pas le savoir. Jean rappelle qu'il y a peut-être un enjeu légal à considérer et qu'il faudrait vérifier si la construction d'une niche est soumise à la réglementation des métiers de la construction.
		
		
		Luc demande si on pourrait faire ce travail nous-même étant donné la petitesse du mandat et qu'il ne s'agit que d'une niche à chien.
		

	\end{enumerate}
\end{enumerate}


% nouveau point
\item \textbf{Points à traiter}

\begin{enumerate}

\item Choix du type de revêtement extérieur

\begin{enumerate}

\item Avantages et inconvénients de chaque type de revêtement

Marie nous a présenté sont tableau synthèse avec les avantages et inconvénients de chaque type de revêtement :

Option 1 : peinture à l'huile

Avantages : économique, relativement simple, nous pourrions faire cette tâche nous-même.

Inconvénients : moins durable que les deux autres options, doit être faite pendant une journée ensoleillée avec aucun risque de pluie, doit être faite en au moins deux étapes à cause de la couche de fond.


Option 2 : revêtement en polymères

Avantages : plus durable que l'option 1, moins cher que l'option 3.

Inconvénients : moins durable que l'option 3, plus cher que l'option 1.


Option 3 : revêtement en acier

Avantages : dure plusieurs décennies.

Inconvénients : la plus chère des 3 options.


\item Prix et durabilité de chaque option

Ce point a déjà été traité.


\end{enumerate}

\item Décision sur la pose du revêtement extérieur
	\begin{enumerate}
		\item Discussion sur les options versus coûts et délais

Comme Luc n'aime pas faire le même travail deux fois, il privilégie une option durable : revêtement en polymères ou en acier.


Marie trouve que le revêtement en acier coûte cher et désirer exclure d'emblée cette option. L'équipe accepte d'exclure l'option du revêtement en acier.


Marie trouve que ce serait plus joli le revêtement en polymères et que ça irait bien avec l'apparence de la maison. Paul propose de déplacer la niche dans la cour arrière si ce n'est qu'une question d'harmonisation d'apparence.


Jean rappelle que l'espérance de vie d'un chien est de 10 à 15 ans et qu'il est donc inutile d'avoir un revêtement trop durable puisque le chien a déjà 5 ans.


Luc ne veut pas avoir à reprendre le travail pour le prochain chien et rappelle que le temps a lui aussi une valeur.


Jean, qui a une formation de base en finances, explique à Luc qu'une solution deux fois plus chère et deux fois plus durable vaut moins que deux solutions deux fois moins chères et deux fois moins durables même si le prix au final est le même. La raison est que l'argent économisée aujourd'hui peut être placée pour générer un rendement.



		\item Option retenue
		
		L'équipe se rend aux arguments de Jean et choisit l'option 1 : la peinture à l'huile. De plus, l'équipe choisit de faire elle-même la peinture.
		
		
		Cette décison rend caduque les mandats de recherche d'informations supplémentaires sur les fournisseurs et la législation dans le domaine de la construction.
		

	\end{enumerate}
\end{enumerate}


% nouveau point
\item \textbf{Divers}

Jean souligne que le présent procès-verbal reprend exactement et textuellement les points de l'ordre du jour.


Marie souligne que si un point a été sauté pour être discuté ultérieurement, il figure quand même dans le procès-verbal, dans les mêmes mots et la même numérotation que dans l'ordre du jour.


Paul rappelle qu'il faut explicitement assigner les tâches de rédaction du procès-verbal, du prochain ordre du jour et de la mise à jour du diagramme de Gantt, par qui et pour quand.


Luc rappelle que toute tâches assignée dans le procès-verbal doit apparaître dans le diagramme de Gantt.

% nouveau point
\item \textbf{Répartition des tâches}

\begin{enumerate}

\item  Ordre du jour, procès-verbal et diagramme de Gantt : Marie enverra au membres de l'équipe un projet d'ordre du jour d'ici mardi, de même que le diagramme de Gantt. Les membres sont invités à communiquer avec Marie d'ici lundi 15h00 pour tout point à ajouter au prochain ordre du jour. Marie s'occupera de la remise des trois fichiers mercredi à 10h00 au plus tard et fera suivre le courriel de confirmation aux autres membres de l'équipe.

\item Achat de la peinture : Paul et Luc (Paul responsable) visiteront d'ici mercredi plusieurs quincailleries pour s'informer sur quelle peinture serait la plus appropriée pour une niche à chien.

\item Achat du matériel : Jean visitera d'ici mercredi plusieurs quincailleries pour s'informer sur le matériel nécessaire pour peindre une niche à chien.

\end{enumerate}


% nouveau point
\item \textbf{Évaluation de la réunion}

La réunion s'est bien déroulée et a permis de préciser certains points. Tous conviennent que ceux qui ont besoin de café devraient l'avoir avec eux avant le début de la réunion pour ne pas avoir à s'absenter pendant la réunion.


% nouveau point
\item \textbf{Date, heure, lieu et objectif de la prochaine réunion}

\begin{tabular}{@{}lll}
   Date: 2018/03/22
   & Heure: 10h30
   &  Lieu: PLT--2708
\end{tabular}
\par
La prochaine réunion a pour objectif de faire un choix sur la peinture à acheter et le matériel à acheter.


% nouveau point
\item \textbf{Fermeture de la réunion}

Heure: 12h25


% nouveau point
\item \textbf{Étaient présents}

\begin{dinglist}{"33}
   \item Marie Marcoux
   \item Luc Laverdière
   \item Paul Paquette
   \item Jean Jubinville
\end{dinglist}

\end{enumerate}

\end{document}
